\def\year{2021}\relax
%File: formatting-instructions-latex-2021.tex
%release 2021.2
\documentclass[letterpaper]{article} % DO NOT CHANGE THIS
\usepackage{aaai21}  % DO NOT CHANGE THIS
\usepackage{times}  % DO NOT CHANGE THIS
\usepackage{helvet} % DO NOT CHANGE THIS
\usepackage{courier}  % DO NOT CHANGE THIS
\usepackage[hyphens]{url}  % DO NOT CHANGE THIS
\usepackage{graphicx} % DO NOT CHANGE THIS
\urlstyle{rm} % DO NOT CHANGE THIS
\def\UrlFont{\rm}  % DO NOT CHANGE THIS
\usepackage{natbib}  % DO NOT CHANGE THIS AND DO NOT ADD ANY OPTIONS TO IT
\usepackage{caption} % DO NOT CHANGE THIS AND DO NOT ADD ANY OPTIONS TO IT
\frenchspacing  % DO NOT CHANGE THIS
\setlength{\pdfpagewidth}{8.5in}  % DO NOT CHANGE THIS
\setlength{\pdfpageheight}{11in}  % DO NOT CHANGE THIS
%\nocopyright
%PDF Info Is REQUIRED.
% For /Author, add all authors within the parentheses, separated by commas. No accents or commands.
% For /Title, add Title in Mixed Case. No accents or commands. Retain the parentheses.
\pdfinfo{
/Title (AAAI Press Formatting Instructions for Authors Using LaTeX -- A Guide)
/Author (AAAI Press Staff, Pater Patel Schneider, Sunil Issar, J. Scott Penberthy, George Ferguson, Hans Guesgen, Francisco Cruz, Marc Pujol-Gonzalez)
/TemplateVersion (2021.2)
} %Leave this
% /Title ()
% Put your actual complete title (no codes, scripts, shortcuts, or LaTeX commands) within the parentheses in mixed case
% Leave the space between \Title and the beginning parenthesis alone
% /Author ()
% Put your actual complete list of authors (no codes, scripts, shortcuts, or LaTeX commands) within the parentheses in mixed case.
% Each author should be only by a comma. If the name contains accents, remove them. If there are any LaTeX commands,
% remove them.

% DISALLOWED PACKAGES
% \usepackage{authblk} -- This package is specifically forbidden
% \usepackage{balance} -- This package is specifically forbidden
% \usepackage{color (if used in text)
% \usepackage{CJK} -- This package is specifically forbidden
% \usepackage{float} -- This package is specifically forbidden
% \usepackage{flushend} -- This package is specifically forbidden
% \usepackage{fontenc} -- This package is specifically forbidden
% \usepackage{fullpage} -- This package is specifically forbidden
% \usepackage{geometry} -- This package is specifically forbidden
% \usepackage{grffile} -- This package is specifically forbidden
% \usepackage{hyperref} -- This package is specifically forbidden
% \usepackage{navigator} -- This package is specifically forbidden
% (or any other package that embeds links such as navigator or hyperref)
% \indentfirst} -- This package is specifically forbidden
% \layout} -- This package is specifically forbidden
% \multicol} -- This package is specifically forbidden
% \nameref} -- This package is specifically forbidden
% \usepackage{savetrees} -- This package is specifically forbidden
% \usepackage{setspace} -- This package is specifically forbidden
% \usepackage{stfloats} -- This package is specifically forbidden
% \usepackage{tabu} -- This package is specifically forbidden
% \usepackage{titlesec} -- This package is specifically forbidden
% \usepackage{tocbibind} -- This package is specifically forbidden
% \usepackage{ulem} -- This package is specifically forbidden
% \usepackage{wrapfig} -- This package is specifically forbidden
% DISALLOWED COMMANDS
% \nocopyright -- Your paper will not be published if you use this command
% \addtolength -- This command may not be used
% \balance -- This command may not be used
% \baselinestretch -- Your paper will not be published if you use this command
% \clearpage -- No page breaks of any kind may be used for the final version of your paper
% \columnsep -- This command may not be used
% \newpage -- No page breaks of any kind may be used for the final version of your paper
% \pagebreak -- No page breaks of any kind may be used for the final version of your paperr
% \pagestyle -- This command may not be used
% \tiny -- This is not an acceptable font size.
% \vspace{- -- No negative value may be used in proximity of a caption, figure, table, section, subsection, subsubsection, or reference
% \vskip{- -- No negative value may be used to alter spacing above or below a caption, figure, table, section, subsection, subsubsection, or reference

\setcounter{secnumdepth}{0} %May be changed to 1 or 2 if section numbers are desired.

% The file aaai21.sty is the style file for AAAI Press
% proceedings, working notes, and technical reports.
%

% Title

% Your title must be in mixed case, not sentence case.
% That means all verbs (including short verbs like be, is, using,and go),
% nouns, adverbs, adjectives should be capitalized, including both words in hyphenated terms, while
% articles, conjunctions, and prepositions are lower case unless they
% directly follow a colon or long dash
\iffalse
\title{AAAI Press Formatting Instructions \\for Authors Using \LaTeX{} --- A Guide }
\author{
    %Authors
    % All authors must be in the same font size and format.
    Written by AAAI Press Staff\textsuperscript{\rm 1}\thanks{With help from the AAAI Publications Committee.}\\
    AAAI Style Contributions by Pater Patel Schneider,
    Sunil Issar,  \\
    J. Scott Penberthy,
    George Ferguson,
    Hans Guesgen,
    Francisco Cruz,
    Marc Pujol-Gonzalez
    \\
}
\affiliations{
    %Afiliations
    \textsuperscript{\rm 1}Association for the Advancement of Artificial Intelligence\\
    %If you have multiple authors and multiple affiliations
    % use superscripts in text and roman font to identify them.
    %For example,

    % Sunil Issar, \textsuperscript{\rm 2}
    % J. Scott Penberthy, \textsuperscript{\rm 3}
    % George Ferguson,\textsuperscript{\rm 4}
    % Hans Guesgen, \textsuperscript{\rm 5}.
    % Note that the comma should be placed BEFORE the superscript for optimum readability

    2275 East Bayshore Road, Suite 160\\
    Palo Alto, California 94303\\
    % email address must be in roman text type, not monospace or sans serif
    publications21@aaai.org

    % See more examples next
}
\fi
\iffalse
%Example, Single Author, ->> remove \iffalse,\fi and place them surrounding AAAI title to use it
\title{My Publication Title --- Single Author}
\author {
    % Author
    Author Name \\
}

\affiliations{
    Affiliation \\
    Affiliation Line 2 \\
    name@example.com
}
\fi


%Example, Multiple Authors, ->> remove \iffalse,\fi and place them surrounding AAAI title to use it
\title{Domain-agnostic Outlier Ranking Algorithms (DORA)---A Configurable Pipeline for Facilitating Outlier Detection in Scientific Datasets}
\author {
    % Authors
    First Author Name,\textsuperscript{\rm 1}
    Second Author Name, \textsuperscript{\rm 2}
    Third Author Name \textsuperscript{\rm 1} \\
}
\affiliations {
    % Affiliations
    \textsuperscript{\rm 1} Affiliation 1 \\
    \textsuperscript{\rm 2} Affiliation 2 \\
    firstAuthor@affiliation1.com, secondAuthor@affilation2.com, thirdAuthor@affiliation1.com
}

\begin{document}

\maketitle

\begin{abstract}
Automatic detection of out-of-distribution (OOD) samples or features is
universally needed when working with scientific datasets. For example,
OOD detection methods can be used for cleaning datasets or flagging
novel samples in order to guide instrument acquisition or scientific analysis.
We present Domain-agnostic Outlier Ranking Algorithms (DORA) to provide
a configurable pipeline that makes it easy for scientists to apply and evaluate
outlier detection methods in a variety of domains. DORA allows users to 
configure outlier detection experiments by specifying the location of their 
dataset(s), the input data type, feature extraction methods, and which 
algorithms should be applied. DORA supports image, raster, time series, 
or feature vector input data types and outlier detection methods that include
 Isolation Forest, Discovery via Eigenbasis Modeling of Uninteresting Data 
 (DEMUD), principal component analysis (PCA), Reed-Xiaoli (RX) detector, 
 Local RX, negative sampling, and probabilistic autoencoder. Given a set of
  data samples, each algorithm assigns an outlier score to each sample. DORA
   provides results organization and visualization modules to help users 
   process the results, including sorting samples by outlier score, evaluating 
   outlier recall for a set of known/labeled outliers, clustering groups of 
   similar outliers together, and causal graphs. To demonstrate the 
   cross-domain applicability of DORA, we conducted outlier detection
    experiments for three real-world use cases: 1) cleaning noisy ground-truth 
    datasets (Earth Science), 
     2) flagging non-astrophysical objects in the Dark Energy Survey 
     (astrophysics), and 3) novelty-guided onboard targeting of instruments
      for Mars rovers (planetary science). [We showed that ...]
\end{abstract}

\section{Introduction}
The ability to automatically detect out-of-distribution samples in large data 
sets is of interest for a wide variety of scientific domains. Depending on the
 application setting, this capability is also commonly referred to as anomaly
  detection, outlier detection, or novelty detection. More broadly, this is 
  referred to as out-of-distribution (OOD) detection. In general, the goal of 
  OOD detection systems is to identify samples that deviate from the majority
   of samples in a dataset in an unsupervised manner (Pimentel et al., 2014). 
   In machine learning, these methods are commonly used for identifying 
   mislabeled or otherwise invalid samples in a dataset 
   (e.g., Liang et al., 2018; Böhm et al., 2020). 
   
   When working with science datasets, OOD detection can be used for 
   cleaning datasets, e.g., flagging ground-truth labels with GPS or human
    entry error or identifying wrongly categorized objects in a catalog. 
    It could also be used for discovery, e.g., to flag novel samples in order 
    to guide instrument acquisition or scientific analysis. Another application
   is the detection of rare objects that are known to exist but the known
   examples are too few to create a large enough labeled dataset for 
   supervised classification algorithms. 
  Despite wide differences in applications, data types, and dimensionality,
 the same underlying machine learning algorithms can be employed across 
 all of these domains. A challenge for applying them however is that domain
 scientists do not always have the programming or machine learning background
 to apply the algorithms themselves using existing tools. Given the widespread 
 applicability and transferability of OOD methods, the scientific community 
 would benefit from a tool that made it easy for them to apply popular outlier
 detection algorithms to their science datasets. We created DORA (Domain-
 agnostic Outlier Ranking Algorithms) to provide a tool for applying outlier 
 detection algorithms to a variety of scientific data sets with minimal coding
 required. Users need only to specify details for their data/application 
 including the data type, location, and algorithms to run in an experiment
 configuration file. DORA supports image, raster, time series, 
or feature vector input data types and outlier detection methods that include
 Isolation Forest, Discovery via Eigenbasis Modeling of Uninteresting Data 
 (DEMUD), principal component analysis (PCA), Reed-Xiaoli (RX) detector, 
 Local RX, negative sampling, and probabilistic autoencoder. Given a set of
  data samples, each algorithm assigns an outlier score to each sample. DORA
   provides results organization and visualization modules to help users 
   process the results, including sorting samples by outlier score, evaluating 
   outlier recall for a set of known/labeled outliers, clustering groups of 
   similar outliers together, and causal graphs. To demonstrate the 
   cross-domain applicability of DORA, we conducted outlier detection
    experiments for three real-world use cases: 1) cleaning noisy ground-truth 
    datasets (Earth Science), 
     2) flagging non-astrophysical objects in the Dark Energy Survey 
     (astrophysics), and 3) novelty-guided onboard targeting of instruments
      for Mars rovers (planetary science). [We showed that ...]

\section{Related Work}
In astrophysics, 
    outlier detection methods have been employed for identifying astrophysical
     objects with unique characteristics (Xiong et al., 2010; 
     Storey-Fisher et al., 2020; Hayat et al., 2020) as well as 
     data/modeling artifacts~\citep{wagstaff:des-anom20} (Lochner et al., 2020) in 
     astronomical surveys. Examples of outlier detection applications in 
     Earth science include detecting anomalous objects or materials 
     (Chang and Chiang, 2002; Zhou et al., 2016), data artifacts/noise 
     (Liu et al., 2017; Alvera-Azcárate et al., 2012), change (Touati et al.,
      2020; Zhou et al., 2016),  and ocean extremes (Prochaska et al.). 
      Planetary science applications of outlier detection have mostly focused 
      on prioritizing samples with novel geologic or geochemical features for
       follow-up targeting or analysis (Kerner et al., 2020a; Kerner et al., 
       2020b) \citep{wagstaff:demud13}.

\begin{itemize}
\item Methods for outlier detection
\item Applications of outlier detection related to our focus domains 
(e.g., \citet{kerner2020comparison})
\item Tools for applying outlier detection or other methods across domains?
\end{itemize}

\section{Methods}

\subsection{DORA configurable pipeline} Steven please add details about the 
approach/design choices for this pipeline. Mention that additional data types, 
feature extraction methods, ranking algorithms, and results organization and 
visualization modules can easily be added and integrated if they follow the API. 
Hannah to add architecture figure. Add descriptions of each layer of modules: 
\paragraph{Data types} Hannah
\paragraph{Feature extraction} Hannah 
\paragraph{Ranking algorithms} We implemented 7 algorithms for scoring and 
ranking samples by outlierness. [Describe each algorithm briefly and how they 
complement each other.]
\paragraph{Results organization} Eric, Umaa, Steven, Vinay
\paragraph{Outlier visualization} Jake?

\section{Experiments}

\subsection{Datasets}
Describe the datasets overall and our motivation for choosing these. Maybe add
 a table with number of samples, number of reference known outliers (if 
 available), etc, if room? 

\subsubsection{Astrophysical objects in Dark Energy Survey}
Umaa

\subsubsection{Targets in Mars rover images}
Kiri

\subsubsection{Satellite time series for ground-truth observations}
Hannah: Describe the Earth crop field data set and applications

% Decided to cut this use case to make the presentation simpler (only one
% Earth science use case, and we already have a 1-band image dataset
% (Navcam). 
%\subsubsection{Volcanic thermal anomalies}

\subsubsection{MNIST and FashionMNIST}
We added this to have a traditional benchmark and use case demonstration. 

\subsection{Results}
Discuss the overall approach to the setup of the experiments. In each 
subsection discuss specific setup details (including evaluation metrics) for 
each dataset then the results from the experiments.

\subsubsection{Astrophysical objects in Dark Energy Survey}
Umaa

\subsubsection{Targets in Mars rover images}
Kiri

\subsubsection{Satellite time series for ground-truth observations}
Hannah: Describe the Earth crop field data set and applications

%\subsubsection{Volcanic thermal anomalies}

\subsubsection{MNIST and FashionMNIST}
We added this to have a traditional benchmark and use case demonstration. 


\section{Discussion}
\begin{itemize}
\item Discuss results from experiments
\item Challenges of evaluating results in outlier detection?
\item Downstream use of results in science applications/Path to deployment
\end{itemize}

\section{Conclusions}
Summarize conclusions and plans for future work, including public 
pip-installable package

[Note: these datasets could also be used by other researchers as OOD
 benchmark tasks?]


\bibliography{dora_references}


\section{ Acknowledgments}
Funding/support: JPL CIF, SMD etc. Acknowledge people who contributed to 
datasets e.g. FAO, Matt Pritchard, DES, other CIF team members?



\end{document}
